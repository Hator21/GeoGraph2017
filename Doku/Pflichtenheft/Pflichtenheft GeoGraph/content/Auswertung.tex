\newpage
	\section{\Large ZIELBESTIMMUNG}
	Das Benutzungsziel ist:
	\begin{itemize}
		\item Es soll möglich sein OSM Ausschnitte über die OSM API abzufragen. Dieser Ausschnitt wird über eine Boundingbox ausgewählt.
		\item Sobald ein Ausschnitt geladen wurde, kann ein Punkt via Koordinate(Längen- und Breitengrad) ausgewählt werden und der nächstliegende Node zu diesem Punkt soll dann zentriert werden.
		\item Der Kartenausschnit soll Verschiebbar, Vergrößerbar und Verkleinerbar sein.
	\end{itemize}
	\subsection{Musskriterien}
	\begin{itemize}
		\item Das System muss auf dem Kartenbezugssystem WGS 84 laufen
		\item Das System muss nach Eingabe von Breiten- \& Längengrad eine Teilkarte ausgeben. Auf dieser Karte sind die Bundesautobahnen und Bundesstraßen sowie Richtungspfeile in die, die Autobahn/Straße verläuft, eingezeichnet. Dabei zeigen die Pfeile in die jeweilige Richtung der nächsten Node.
		\item Das System muss die Pfeile, so anpassen das die Länge der Pfeile in proportionaler abhängig zur Geschwindigkeitsbeschränkung stehen. 
		\item Das System muss nach Eingabe einer minimalen und maximalen-Eingabe eines Punkten, den Ausschnitt der Karte darstellen.
		\item Das System muss nachdem eine Karte dargestellt wurde, den ausgewählten Kartenbereich verschieben können.
		\item Das System muss nach laden eines Kartenbereichs diesen Verschieben, Vergrößern, und Verkleinern können.
		\item Das System muss skalierbar sein.
	\end{itemize}
	\subsection{Abgrenzungskriterien}
	\begin{itemize}
		\item Das System ist keine Navigations Software.
	\end{itemize}
	
	
	\section{\Large PRODUKTEINSATZ}
	\subsection{Anwendungsbereiche}
	\begin{itemize}
		\item Das Produkt soll im privaten Bereich eines Benutzers Anwendung finden.\\
		Es soll nicht für gewerbliche Zwecke oder für Anbahnung von Geschäften genutzt werden.
	\end{itemize}
	\subsection{Zielgruppen}
	\begin{itemize}
		\item Die Zielgruppe sind Leute, 
		\begin{itemize}
			\item wie Herr Dr. Fünfzig
			\item die Wert auf \textbf{"Wege zur Gewinnung und Korrektur von Kartendaten"} legen.(Aus Anfordernungen des Kunden entnommen)
			\item die Initiativen für \textbf{"GeoInformation und Navigation"} unterstützen.
		\end{itemize}
	\end{itemize}
	\subsection{Betriebsbedingungen}
	\begin{itemize}
		\item Das Produkt benötigt eine stetige Internetverbindung und den Dienst der die *.OSM Dateien zur Verfügung stellt. Unser Service wird angeboten solange wir Zugriff auf die *.OSM Dateien haben.
	\end{itemize}
	
	
	\section{\Large PRODUKTÜBERSICHT}
	Gibt eine Übersicht über das Produkt, z.B. über alle wichtigen Geschäftsprozesse in Form eines Übersichtsdiagramms.
	\subsection{Usecase Diagramm}
		\begin{figure}[H]
			\centering
			\includegraphics[width=0.7\linewidth]{images/Usecases}
			\caption{UseCase Diagramm}
			\label{fig:Usecase Diagramm}
		\end{figure}
		
		
	\section{\Large PRODUKTFUNKTIONEN}
	\subsection{Usecase-Beschreibungen}

	\begin{table}[H]
		\begin{tabular}{|p{8cm}|p{8cm}|}
			\hline
			\textbf{GEO-01 } \\ 
			\hline
			\textbf{ID :}\centering & GEO-01  \\ \hline 
			\textbf{Title :}\centering & Abruf der Daten über API \\ \hline 
			\textbf{Description :}\centering & Daten für die Karte werden per API abgerufen \\ \hline 
			\textbf{Trigger :}\centering & User klickt auf den Button "Nach Koordinaten suchen" \\ \hline 
			\textbf{Primary Actor :} \centering & User \\ \hline 
			\textbf{Preconditions :}\centering & 
			1. Programm ist gestartet \newline 
			2. User hat Längen und Breitengrad eingegeben	\\ \hline 
			\textbf{Postconditions :}\centering &  
			1. User hat den Kartenbereich erfolgreich geladen \newline 
			2. Karte wird angezeigt \\ \hline
			\textbf{Other Use Cases :}\centering & - \\ \hline  
			\textbf{Main Success Scenario :}\centering & 
			1. User gibt Längen und Breitengrad ein \newline
			2. User klickt auf "Nach Kordinaten suchen" \newline
			3. Karte wird angezeigt \\ \hline  
			\textbf{Extensions :}\centering & - \\ \hline  
			\textbf{Priority :}\centering & High \\ \hline  
		\end{tabular}
	\end{table}
	
	\begin{table}[H]
		\begin{tabular}{|p{8cm}|p{8cm}|}
			\hline
			\textbf{GEO-02 } \\ 
			\hline
			\textbf{ID :}\centering & GEO-02  \\ \hline 
			\textbf{Title :}\centering & Laden der Daten aus einer OSM-Datei \\ \hline 
			\textbf{Description :}\centering & Daten für die Karte werden aus der hinterlegten OSM-Datei geladen \\ \hline 
			\textbf{Trigger :}\centering & User klickt auf den Button "Nach Koordinaten suchen" \\ \hline 
			\textbf{Primary Actor :} \centering & User \\ \hline 
			\textbf{Preconditions :}\centering & 
			1. GEO-01\\ \hline 
			\textbf{Postconditions :}\centering & 
			1. User hat Kartenbereich aus OSM-Datei geladen \newline
			2. Karte wird angezeigt \\ \hline
			\textbf{Other Use Cases :}\centering & - \\ \hline  
			\textbf{Main Success Scenario :}\centering & 
			1. GEO-01 \newline
			2. User klickt auf " Nach Kordinaten suchen" \newline
			3. Karte wird angezeigt \\ \hline  
			\textbf{Extensions :}\centering & - \\ \hline  
			\textbf{Priority :}\centering & High \\ \hline  
		\end{tabular}
	\end{table}
	
	\begin{table}[H]
		\begin{tabular}{|p{8cm}|p{8cm}|}
			\hline
			\textbf{GEO-03 } \\ 
			\hline
			\textbf{ID :}\centering & GEO-03  \\ \hline 
			\textbf{Title :}\centering & Skalierung des Kartenbereichs  \\ \hline 
			\textbf{Description :}\centering & Skaliert den Kartenbereich via Regler  \\ \hline 
			\textbf{Trigger :}\centering & User bewegt den Slider in den positiven/negativen Bereich  \\ \hline 
			\textbf{Primary Actor :} \centering & User \\ \hline 
			\textbf{Preconditions :}\centering & 
			1. User hat GEO-01 oder GEO-02 ausgeführt \\ \hline 
			\textbf{Postconditions :}\centering & 
			1. User bewegt Slider in den positiven/negativen Bereich \newline
			2. Kartenausschnitt vergrößert/verkleinert sich \newline
			3. Karte wird angezeigt \\ \hline
			\textbf{Other Use Cases :}\centering & - \\ \hline  
			\textbf{Main Success Scenario :}\centering & 
			1. GEO-01 oder GEO-02 \newline
			2. User bewegt Slider in Positiven/Negativen Bereich \newline
			3. Karte wird vergrößert/verkleinert \newline
			4. Karte wird angezeigt \\ \hline  
			\textbf{Extensions :}\centering & - \\ \hline  
			\textbf{Priority :}\centering & High \\ \hline  
		\end{tabular}
	\end{table}
	
	\begin{table}[H]
		\begin{tabular}{|p{8cm}|p{8cm}|}
			\hline
			\textbf{GEO-04 } \\ 
			\hline
			\textbf{ID :}\centering & GEO-04  \\ \hline 
			\textbf{Title :}\centering & Verschiebung des Kartenbereichs \\ \hline 
			\textbf{Description :}\centering & Verschiebt den Kartenbereich via Maus \\ \hline 
			\textbf{Trigger :}\centering & User bewegt die Maus in den Kartenausschnitt und hält die linke Maustaste gedrückt und schiebt dann in x/y Richtung  \\ \hline 
			\textbf{Primary Actor :} \centering & User \\ \hline 
			\textbf{Preconditions :}\centering & 
			1. GEO-01 oder GEO-02 \\ \hline 
			\textbf{Postconditions :}\centering & 
			1. User bewegt die Maus in x/y Richtung \newline
			2. Der Kartenausschnitt bewegt sich in x/y Richtung \newline
			3. Der Kartenausschnitt wird angezeigt \\ \hline
			\textbf{Other Use Cases :}\centering & - \\ \hline  
			\textbf{Main Success Scenario :}\centering & 
			1. GEO-01 oder GEO-02 \newline
			2. User hält Maus gedrückt und schiebt den Kartenausschnitt \newline
			3. Kartenausschnitt wird angezeigt \\ \hline  
			\textbf{Extensions :}\centering & 
			1. Nur zuvor geladener Kartenausschnitt wird angezeigt \\ \hline  
			\textbf{Priority :}\centering & High \\ \hline  
		\end{tabular}
	\end{table}
	
	\begin{table}[H]
		\begin{tabular}{|p{8cm}|p{8cm}|}
			\hline
			\textbf{GEO-05 } \\ 
			\hline
			\textbf{ID :}\centering & GEO-05  \\ \hline 
			\textbf{Title :}\centering & Zentrierung auf einer Node \\ \hline 
			\textbf{Description :}\centering & Zentrierung auf einer Node nach eingabe von Langen-und Breitengrad \\ \hline 
			\textbf{Trigger :}\centering & User gibt Breiten-und Längengrad ein und die nächstgelegende Node wird zentriert \\ \hline 
			\textbf{Primary Actor :} \centering & User \\ \hline 
			\textbf{Preconditions :}\centering & 
			1. GEO-01 oder GEO-02 \\ \hline 
			\textbf{Postconditions :}\centering &
			1. Kartenausschnitt wird auf die nächstgelgende Node verschoben \newline
			2. Karte wird auf die Node zentriert \newline
			3. Karte wird angezeigt \\ \hline
			\textbf{Other Use Cases :}\centering & - \\ \hline  
			\textbf{Main Success Scenario :}\centering &
			1. GEO-01 oder GEO-02 \newline
			2. Kartenausschnitt wird verschoben \newline
			3. Karte wird auf Node zentriert \newline
			4. Karte wird angezeigt \\ \hline  
			\textbf{Extensions :}\centering & - \\ \hline  
			\textbf{Priority :}\centering & High \\ \hline  
		\end{tabular}
	\end{table}	
	
	\subsection{Aktivitätsdiagramm}
		\begin{figure}[H]
			\centering
			\includegraphics[width=0.7\linewidth]{images/Ablauf}
			\caption{Aktivitäts Diagramm}
			\label{fig:Aktivitäts Diagramm}
		\end{figure}
	\subsection{Sequenzdiagramm}
	\begin{figure}[H]
	\centering
	\includegraphics[width=0.7\linewidth]{images/Squenz}
	\caption{Sequenz Diagramm}
	\label{fig:Sequenz Diagramm}
	\end{figure}
	
\newpage
	\section{\Large PRODUKTDATEN}
		Langfristig sollen folgende Daten im System gespeichert | ausgelesen werden:
	\begin{itemize}
		\item Speicherung der OSM-Datei in folgendem Format: 
		\begin{itemize}
			\item Min und Max der BoundingBox
			\item 51.9\_52.1\_52.1\_53.0.osm (Beispiel)
		\end{itemize}
		\item Laden der Daten via Overpass API
	\end{itemize}
	\subsection{Analyseklassendiagramm}	
		\begin{figure}[H]
		\centering
		\includegraphics[width=0.7\linewidth]{images/Klassendiagramm}
		\caption{Klassendiagramm}
		\label{fig:Klassendiagramm}
	\end{figure}
	\subsection{Paketdiagramm}
		\begin{figure}[H]
			\centering
			\includegraphics[width=0.7\linewidth]{images/Paketdiagramm}
			\caption{Paketdiagramm}
			\label{fig:Paketdiagramm}
		\end{figure}
	\subsection{Domänenklassendiagramm}
		\begin{figure}[H]
			\centering
			\includegraphics[width=0.7\linewidth]{images/DomaenenKlassendiagramm}
			\caption{Domänenklassendiagramm}
			\label{fig:Domänenklassendiagramm}
		\end{figure}
	\section{\Large PRODUKTLEISTUNGEN}
	\begin{itemize}
		\item Nicht genauer spezifiziert.
	\end{itemize} 
		
	
	\section{\Large QUALITÄTSANFORDERUNGEN}
	\begin{itemize}
		\item Nicht genauer spezifiziert.
	\end{itemize}
	
	\section{\Large BENUTZEROBERFLÄCHE}
	Es gibt eine Rolle und das ist die des Users der das Prgoramm ausführt (GUI).
	\begin{figure}[H]
	\centering
	\includegraphics[width=0.7\linewidth]{images/OSMStreetView}
	\caption{Benutzeroberfläche}
	\label{fig:GUI}
	\end{figure}
	
	\begin{figure}[H]
	\centering
	\includegraphics[width=0.7\linewidth]{images/OSMStreetView2}
	\caption{Benutzeroberfläche}
	\label{fig:GUI2}
	\end{figure}
	
	\subsection{Zustandsdiagramme}
	\begin{figure}[H]
	\centering
	\includegraphics[width=0.7\linewidth]{images/Zustandsdiagramm}
	\caption{Zustands Diagramm}
	\label{fig:Zustands Diagramm}
	\end{figure}
	
	
	\section{\Large NICHTFUNKTIONALE ANFORDERUNGEN}
	Es werden alle Anforderungen aufgeführt, die sich nicht auf die Funktionalität, \textbf{ die Leistung} und \textbf{ die Benutzungsoberfläche} beziehen, z.B. :
	\begin{itemize}
		\item Einzuhaltende \textbf{Gesetze}
		\item Einzuhaltende \textbf{Normen}
		\item Testat durch externe Prüfungsgesellschaft
		Revisionsfähigkeit 
		\item Ordnungsmäßigkeit der Buchführung
		\item \textbf{ Sicherheitsanforderungen, z.B. :}
		\begin{itemize}
			\item Richtigkeit der Nodes
			\item Richtigkeit der Pfeile
			\item Genaues Darstellen der Nodes in abhängigkeit zur OSM-Datei
			\item Genauigkeit der BoundingBox
			\item Genauigkeit beim Skalieren
		\end{itemize}  
		\item Plattformabhängigkeiten
		\item Performant in Abhängigkeit zur Downloadgeschwindigkeit und API
		\item Wenn der markierte Bereich der Boundingbox zu groß ist, dann kann das laden der Nodes sehr lange dauern
		\item Aktuelle Betriebssysteme abdecken(Windows, Linux)
		\item Abgefragte OSM-Datein werden lokal gespeichert mit den Min und Max Angaben der BoundingBox	(bsp. 51.9\_52.1\_52.1\_53.0.osm) 
	\end{itemize} 

	
	\section{\Large TECHNISCHE PRODUKTUMGEBUNG}
   	In diesem Kapitel wird die technische Umgebung des Produkts beschrieben.\\
   	Bei Client / Server-Anwendungen ist die Umgebung jeweils für Clients und Server getrennt anzugeben.
	\subsection{Software}
	\begin{itemize}
		\item Erfordert \textbf{Java 8.x} auf dem Client
		\begin{itemize}
			\item getestet und entworfen wird für :
			\begin{itemize}
				\item PC | Laptop
				\begin{itemize}
					\item Windows ab Version 7
					\item Linux
				\end{itemize}
			\end{itemize}
		\end{itemize}
	\end{itemize}
	\subsection{Hardware}
	\begin{itemize}
		\item \textbf{Internetfähiges Gerät :}
		\begin{itemize}
			\item PC | Laptop
			\item \textbf{Minimale Bildschirmauflösung :}
				\begin{itemize}
					\item 1024 x 768 Pixel Hochformat / Querformat
				\end{itemize}
					\item \textbf{Maximale Bildschirmauflösung :}
				\begin{itemize}
				\item 4096 × 2160 Pixel Hochformat / Querformat
				\end{itemize}		
		\end{itemize}
	\end{itemize}
	\subsection{Orgware}
	\begin{itemize}
		\item Der Client benötigt eine Internetverbindung.
		\item Um eine befriedigende Nutzererfahrung zu gewährleisten, werden folgende Bandbreiten-Untergrenzen definiert:
		\begin{itemize}
			\item \textbf{ PC | Laptop :}
			\begin{itemize}
				\item DSL Verbindung mit min. 2 Mbit/s Download-Bandbreite
			\end{itemize}
		\end{itemize}
	\end{itemize}
	\subsection{Produkt-Schnittstellen}
	
	
	\section{\Large SPEZIELLE ANFORDERUNGEN AN DIE ENTWICKLUNGS-UMGEBUNG}
	Entwicklung- und Testumgebung des Frontends: Siehe 10 Technische Produktentwicklung 
	\subsection{Software}
	\subsection{Hardware}
	\subsection{Orgware}
	\subsection{Entwicklungsschnittstellen}
	
	
	\section{\Large GLIEDERUNG IN TEILPRODUKTE}
	
	
	\section{\Large ERGÄNZUNGEN}
	Ein erster Testbetrieb wird in der Arbeitsumgebung des Kunden stattfinden. Dort wird dann zunächst ausgiebig die Stabilität und Sicherheit des Systems getestet.
	
\newpage
	\section{\Large GLOSSAR}
	In diesem Kapitel wird die spezifische Sprache des Auftraggebers wie \textbf{ Kürzel } und \textbf{ Fachbegriffe } beschrieben, z.B. :
	\begin{itemize}
		\item \textbf{ System }
		\begin{itemize}
			\item Hiermit ist das gesamte Programm gemeint 
		\end{itemize}
		\item \textbf{ skalierbar}
		\begin{itemize}
			\item in Bezug auf das System, ist hier gemeint das es erweiterbar sein soll
		\end{itemize}
		\item \textbf{ User }
		\begin{itemize}
			\item Bearbeitet das Programm
		\end{itemize}
		\item \textbf{ Pfeile }
		\begin{itemize}
			\item zeigen auf den nächsten Node der Straße
			\item länge abhängig zur Skalierung und der Geschwindigkeitsbeschränkung
		\end{itemize}
		\item \textbf{ BoundingBox }
		\begin{itemize}
			\item Auschnitt des Kartenbereichs
			\item Besteht aus Min- und Max-Wert
		\end{itemize}
		\item \textbf{ Node }
		\begin{itemize}
			\item Ein Node stellt einen Punkt auf der Karte dar
			\item Mehrere Nodes können z.b. einer Straße zusammen gefasst werden
		\end{itemize}
		\item \textbf{ OSM-Datei }
		\begin{itemize}
			\item Beinhaltet die Karteninformationen in Form von Nodes 
		\end{itemize}
		\item \textbf{ Java }
		\begin{itemize}
			\item Eine Platform unabhängige Programmiersprache 
		\end{itemize}
		\item \textbf{ API }
		\begin{itemize}
			\item Eine Schnittstelle die über das HTTP-Protokoll angesprochen werden kann 
		\end{itemize}
	\end{itemize}

		