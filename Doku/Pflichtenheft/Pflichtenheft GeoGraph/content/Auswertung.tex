\newpage
	\section{\Large ZIELBESTIMMUNG}
	\begin{itemize}
		\item Korrektheit der Nodes
		\item Struktur
		\item Benutzerfreundlichkeit
	\end{itemize}
	\subsection{Musskriterien}
	\begin{itemize}
		\item Das System muss auf dem Kartenbezugssystem WGS 84 laufen
		\item Das System muss nach Eingabe von Breiten- \& Längengrad eine Teilkarte ausgeben. Auf dieser Karte sind die Bundesautobahnen und Bundesstraßen sowie Richtungspfeile in die, die Autobahn/Straße verläuft, eingezeichnet. Dabei zeigen die Pfeile in die jeweilige Richtung der nächsten Node.
		\item Das System muss die Pfeil-Nodes, so anpassen das z.b. Geschwindigkeitsbeschränkungen gespeichert werden können. 
		\item Das System muss skalierbar sein.
	\end{itemize}
	\subsection{Abgrenzungskriterien}
	\begin{itemize}
		\item Das System ist keine Navigations Software
	\end{itemize}
	
	\section{\Large PRODUKTEINSATZ}
	\subsection{Anwendungsbereiche}
	\begin{itemize}
		\item Das Produkt soll im privaten Bereich eines Benutzers Anwendung finden.\\
		Es soll nicht für gewerbliche Zwecke oder für Anbahnung von Geschäften genutzt werden.
	\end{itemize}
	\subsection{Zielgruppen}
	\begin{itemize}
		\item Die Zielgruppe sind Leute, 
		\begin{itemize}
			\item die Wert auf \textbf{"Wege zur Gewinnung und Korrektur von Kartendaten"} legen.
			\item die Initiativen für \textbf{"GeoInformation und Navigation"} unterstützen.
		\end{itemize}
	\end{itemize}
	\subsection{Betriebsbedingungen}
	\begin{itemize}
		\item Das Produkt benötigt eine stetige Internetverbindung und den Dienst der die *.OSM Dateien zur Verfügung stellt. Unser Service wird angeboten solange wir zugriff auf die *.OSM Dateien haben.
	\end{itemize}
	
	
	\section{\Large PRODUKTÜBERSICHT}
	Gibt eine Übersicht über das Produkt, z.B. über alle wichtigen Geschäftsprozesse in Form eines Übersichtsdiagramms.
%	\subsection{Usecase Anzeige der Karten Informationen über die OSM API}
%	\subsection{Usecase Anzeige der Karten Informationen über die OSM Datei}
	\subsection{Usecase Diagramm}
	\begin{figure}[H]
	\centering
	\includegraphics[width=0.7\linewidth]{images/Usecases}
	\caption{}
	\label{fig:GUI}
	\end{figure}
	\section{\Large PRODUKTFUNKTIONEN}
	\subsection{Usecase-Beschreibungen}

	\begin{table}[H]
		\begin{tabular}{|p{8cm}|p{8cm}|}
			\hline
			\textbf{GEO-01 } \\ 
			\hline
			\textbf{ID :}\centering & GEO-01  \\ \hline 
			\textbf{Title :}\centering & Abruf der Daten über API \\ \hline 
			\textbf{Description :}\centering & Daten für die Karte werden per API abgerufen \\ \hline 
			\textbf{Trigger :}\centering & Klick auf den Knopf '"Nach Koordinaten suchen'" \\ \hline 
			\textbf{Primary Actor :} \centering & User \\ \hline 
			\textbf{Preconditions :}\centering & Internetverbindung\\ \hline 
			\textbf{Postconditions :}\centering & - \\ \hline
			\textbf{Other Use Cases :}\centering & - \\ \hline  
			\textbf{Main Success Scenario :}\centering & Karte wird angezeigt \\ \hline  
			\textbf{Extensions :}\centering & - \\ \hline  
			\textbf{Priority :}\centering & High \\ \hline  
		\end{tabular}
	\end{table}
	\begin{table}[H]
		\begin{tabular}{|p{8cm}|p{8cm}|}
			\hline
			\textbf{GEO-02 } \\ 
			\hline
			\textbf{ID :}\centering & GEO-02  \\ \hline 
			\textbf{Title :}\centering & Abruf der Daten aus einer OSM-Datei \\ \hline 
			\textbf{Description :}\centering & Daten für die Karte werden aus der hinterlegten OSM-Datei abgerufen \\ \hline 
			\textbf{Trigger :}\centering & Klick auf den Knopf '"Nach Koordinaten suchen'" \\ \hline 
			\textbf{Primary Actor :} \centering & User \\ \hline 
			\textbf{Preconditions :}\centering & Bei Maximalen angaben\\ \hline 
			\textbf{Postconditions :}\centering & - \\ \hline
			\textbf{Other Use Cases :}\centering & - \\ \hline  
			\textbf{Main Success Scenario :}\centering & Karte wird angezeigt \\ \hline  
			\textbf{Extensions :}\centering & - \\ \hline  
			\textbf{Priority :}\centering & High \\ \hline  
		\end{tabular}
	\end{table}
	\subsection{Aktivitätsdiagramm}
	\begin{figure}[H]
	\centering
	\includegraphics[width=0.7\linewidth]{images/Ablauf}
	\caption{}
	\label{fig:Akitiviti}
	\end{figure}
	\subsection{Sequenzdiagramm}
	\begin{figure}[H]
	\centering
	\includegraphics[width=0.7\linewidth]{images/Squenz}
	\caption{}
	\label{fig:Seq}
	\end{figure}
	
	
	\section{\Large PRODUKTDATEN}
		Langfristig sollen folgende Daten im System gespeichert | ausgelesen werden:
	\begin{itemize}
		\item Speicherung der Straßenpunkte als OSM-Datei
		\item Laden der Daten via Overpass API
	\end{itemize}
%	\subsection{Analyseklassendiagramm}
%	\subsection{Klassendiagramm GeoGraph 2017}
	\section{\Large PRODUKTLEISTUNGEN}
	\begin{itemize}
		\item Nicht genauer spezifiziert.
	\end{itemize} 
		
	
	\section{\Large QUALITÄTSANFORDERUNGEN}
	\begin{itemize}
		\item Nicht genauer spezifiziert.
	\end{itemize}
	
	\section{\Large BENUTZEROBERFLÄCHE}
	Es gibt nur eine Rolle und das ist die des Admins, der das Prgoramm ausführt (GUI).
	\begin{figure}[H]
	\centering
	\includegraphics[width=0.7\linewidth]{images/GUI}
	\caption{}
	\label{fig:GUI}
	\end{figure}
%	\subsection{Ansicht: verschiedene Ansichten 1 2 3 4 5}
%	\subsection{Zusatandsdiagramme}
%	\subsection{Ansicht: verschiedene Zustandsdiagramme 1 2 3 4 5}
	
	
	\section{\Large NICHTFUNKTIONALE ANFORDERUNGEN}
	Es werden alle Anforderungen aufgeführt, die sich nicht auf die Funktionalität, \textbf{ die Leistung} und \textbf{ die Benutzungsoberfläche} beziehen, z.B. :
	\begin{itemize}
		\item Einzuhaltende \textbf{Gesetze}
		\item Einzuhaltende \textbf{Normen}
		\item Testat durch externe Prüfungsgesellschaft
		Revisionsfähigkeit 
		\item Ordnungsmäßigkeit der Buchführung
		\item \textbf{ Sicherheitsanforderungen, z.B. :}
		\begin{itemize}
			\item Richtigkeit der Nodes
			\item Richtigkeite der Pfeil-Nodes
			\item Mitlaufen von Protokollen
			\item Sichere Übertragung
		\end{itemize}  
		\item Plattformabhängigkeiten
		\item Sehr performant
		\item Aktuelle Betriebssysteme abdecken
		\item Datenkommunikation über einen verschlüsselten Weg
		\item Daten müssen gespeichert werden	 
	\end{itemize} 

	
	\section{\Large TECHNISCHE PRODUKTUMGEBUNG}
   	In diesem Kapitel wird die technische Umgebung des Produkts beschrieben.\\
   	Bei Client / Server-Anwendungen ist die Umgebung jeweils für Clients und Server getrennt anzugeben.
	\subsection{Software}
	\begin{itemize}
		\item - Erfordert Java auf dem Client
		\begin{itemize}
			\item getestet und entworfen wird für :
			\begin{itemize}
				\item PC | Laptop
				\begin{itemize}
					\item Windows ab Version 7
					\item Linux
				\end{itemize}
			\end{itemize}
		\end{itemize}
	\end{itemize}
	\subsection{Hardware}
	\begin{itemize}
		\item \textbf{Internetfähiges Gerät :}
		\begin{itemize}
			\item PC | Laptop
			\item \textbf{Minimale Bildschirmauflösung :}
				\begin{itemize}
					\item 1024 x 768 Pixel Hochformat / Querformat
				\end{itemize}
					\item \textbf{Maximale Bildschirmauflösung :}
				\begin{itemize}
				\item 4096 × 2160 Pixel Hochformat / Querformat
				\end{itemize}		
		\end{itemize}
	\end{itemize}
	\subsection{Orgware}
	\begin{itemize}
		\item Der Client benötigt eine Internetverbindung.
		\item Um eine befriedigende Nutzererfahrung zu gewährleisten, werden folgende Bandbreiten-Untergrenzen definiert:
		\begin{itemize}
			\item \textbf{ PC | Laptop :}
			\begin{itemize}
				\item DSL Verbindung mit min. 2 Mbit/s Download-Bandbreite
			\end{itemize}
		\end{itemize}
	\end{itemize}
%	\subsection{Produkt-Schnittstellen}
	
	
%	\section{\Large SPEZIELLE ANFORDERUNGEN AN DIE ENTWICKLUNGS-UMGEBUNG}
%	Entwicklung- und Testumgebung des Frontends: Siehe 10 Technische Produktentwicklung 
%	\subsection{Software}
%	\subsection{Hardware}
%	\subsection{Orgware}
%	\subsection{Entwicklungsschnittstellen}
	
	
%	\section{\Large GLIEDERUNG IN TEILPRODUKTE}
	
	
	\section{\Large ERGÄNZUNGEN}
	Ein erster Testbetrieb wird in einer virtuellen Umgebung stattfinden. Dort wird dann zunächst ausgiebig die Stabilität und Sicherheit des Systems getestet.
	
	
	\section{\Large GLOSSAR}
	In diesem Kapitel wird die spezifische Sprache des Auftraggebers wie \textbf{ Kürzel } und \textbf{ Fachbegriffe } beschrieben, z.B. :
	\begin{itemize}
		\item \textbf{ Administrator }
		\begin{itemize}
			\item Bearbeitet das Programm
		\end{itemize}
	\end{itemize}
	\begin{itemize}
		\item \textbf{ Node }
		\begin{itemize}
			\item Eine Node ist eine Kombination aus Punktdaten
		\end{itemize}
	\end{itemize}
	\begin{itemize}
		\item \textbf{ etc. }
		\begin{itemize}
			\item mehr kommt noch ...
		\end{itemize}
	\end{itemize}

		