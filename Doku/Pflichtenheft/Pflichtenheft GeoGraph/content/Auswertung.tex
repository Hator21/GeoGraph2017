\newpage
	\section{\Large ZIELBESTIMMUNG}
	\begin{itemize}
		\item Korrektheit der Nodes
		\item Struktur
		\item Benutzerfreundlichkeit
	\end{itemize}
	\subsection{Musskriterien}
	\begin{itemize}
		\item Das System muss ...
		\item Das System muss ...
	\end{itemize}
	\subsection{Wunschkriterien}
	\begin{itemize}
		\item Das System sollte
	\end{itemize}
	\subsection{Abgrenzungskriterien}
	\begin{itemize}
		\item Was das System nicht ist
	\end{itemize}
	
	\section{\Large PRODUKTEINSATZ}
	\subsection{Anwendungsbereiche}
	\begin{itemize}
		\item Das Produkt soll im privaten Bereich eines Benutzers Anwendung finden.\\
		Es soll nicht für gewerbliche Zwecke oder für Anbahnung von Geschäften genutzt werden.
	\end{itemize}
	\subsection{Zielgruppen}
	\begin{itemize}
		\item Zielgruppe für das GeoGraph Projekt sind Leute die ....
	\end{itemize}
	\subsection{Betriebsbedingungen}
	\begin{itemize}
		\item Das Produkt ...
	\end{itemize}
	
	
	\section{\Large PRODUKTÜBERSICHT}
	\subsection{Usecase: GeoGraph 2017}
	\subsection{Usecases 1 2 3}
	\subsection{Aktivitätsdiagramme}
	\subsection{diagramme 1 2 3}


	\section{\Large PRODUKTFUNKTIONEN}
	\subsection{Usecase-Beschreibungen}

	\begin{table}
		\begin{tabular}{|p{8cm}|p{8cm}|}
			\hline
			\textbf{ODP-01 ...} \\ 
			\hline
			\textbf{ID :}\centering & ODP-01 ... \\ \hline 
			\textbf{Title :}\centering & bla \\ \hline 
			\textbf{Description :}\centering & bla \\ \hline 
			\textbf{Trigger :}\centering & bla \\ \hline 
			\textbf{Primary Actor :} \centering & bla \\ \hline 
			\textbf{Preconditions :}\centering & bla \\ \hline 
			\textbf{Postconditions :}\centering & bla \\ \hline
			\textbf{Other Use Cases :}\centering & bla \\ \hline  
			\textbf{Main Success Scenario :}\centering & bla \\ \hline  
			\textbf{Extensions :}\centering & bla \\ \hline  
			\textbf{Owner :}\centering & bla \\ \hline  
			\textbf{Priority :}\centering & bla \\ \hline  
		\end{tabular}
	\end{table}
	
	\begin{table}
		\begin{tabular}{|p{8cm}|p{8cm}|}
			\hline
			\textbf{ODP-02 ...} \\ 				\hline
			\textbf{ID :}\centering & ODP-02 ... \\ \hline 
			\textbf{Title :}\centering & bla \\ \hline 
			\textbf{Description :}\centering & bla \\ \hline 
			\textbf{Trigger :}\centering & bla \\ \hline 
			\textbf{Primary Actor :} \centering & bla \\ \hline 
			\textbf{Preconditions :}\centering & bla \\ \hline 
			\textbf{Postconditions :}\centering & bla \\ \hline
			\textbf{Other Use Cases :}\centering & bla \\ \hline  
			\textbf{Main Success Scenario :}\centering & bla \\ \hline  
			\textbf{Extensions :}\centering & bla \\ \hline  
			\textbf{Owner :}\centering & bla \\ \hline  
			\textbf{Priority :}\centering & bla \\ \hline  
		\end{tabular}
	\end{table}
	
	\subsection{Sequenzdiagramme}
	\begin{itemize}
		\item blabla
	\end{itemize}
	\subsection{Sequenzdiagramme 1 2 3}
	
	
	\section{\Large PRODUKTDATEN}
		Langfristig sollen folgende Daten im System  gespeichert sein:
	\begin{itemize}
		\item Nodes
		\item ...
		\item ...
	\end{itemize}
		Das Datenaufkommen nimmt im Zeitverlauf zu .... 
	\subsection{Analyseklassendiagramm}
	\subsection{Klassendiagramm GeoGraph 2017}
	
	
	\section{\Large ProduktLeistungen}
	\begin{itemize}
		\item BLABLA
	\end{itemize}	
	
	
	\section{\Large QUALITÄTSANFORDERUNGEN}
	\begin{itemize}
		\item Nicht genauer spezifiziert.
	\end{itemize}
	
	\section{\Large BENUTZEROBERFLÄCHE}
	Nur eine Rolle verfügbar ADMIN
	\subsection{Ansicht: verschiedene Ansichten 1 2 3 4 5}
	\subsection{Zusatandsdiagramme}
	
	
	\section{\Large NICHTFUNKTIONALE ANFORDERUNGEN}
	Es werden alle Anforderungen aufgeführt, die sich nicht auf die Funktionalität, \textbf{ die Leistung} und \textbf{ die Benutzungsoberfläche} beziehen, z.B. :
	\begin{itemize}
		\item Einzuhaltende Gesetze
		\item Einzuhaltende Normen
		\item Testat durch externe Prüfungsgesellschaft
		Revisionsfähigkeit 
		\item Ordnungsmäßigkeit der Buchführung
		\item \textbf{ Sicherheitsanforderungen, z.B. :}
		\begin{itemize}
			\item Passwortschutz
			\item Mitlaufen von Protokollen
			\item Sichere Übertragung
		\end{itemize}  
		\item Plattformabhängigkeiten
		\item Sehr performant
		\item Aktuelle Betriebssysteme abdecken
		\item Datenkommunikation über einen verschlüsselten Weg
		\item Daten müssen gespeichert werden
		 
	\end{itemize} 

	
	\section{\Large TECHNISCHE PRODUKTUMGEBUNG}
   	In diesem Kapitel wird die technische Umgebung des Produkts beschrieben.\\
   	Bei Client/ Server-Anwendungen ist die Umgebung jeweils für Clients und Server getrennt anzugeben.
	\subsection{Software}
	\begin{itemize}
		\item - Erfordert Web-Browser auf dem Client
		
		getestet und entworfen wird für:
		
		- Desktop:
		Mozilla Firefox 35-42, Google Chrome 40-47, Internet Explorer 11
		
		- Android:
		Chrome 40-47, Samsung Browser
		
		- iOS:
		Safari der iOS Versionen 8.x und 9.x (Tablet und iPhone Version)
	\end{itemize}
	\subsection{Hardware}
	\begin{itemize}
		\item \textbf{Internetfähiges Gerät :}
		\begin{itemize}
			\item Desktop PC,  
			\item Tablet,
			\item Sonstige,
		\end{itemize}
		auf dem Java-Code ausführbar ist. 
		\begin{itemize}
			\item \textbf{Minimale Bildschirmauflösung :}
			\begin{itemize}
				\item 1024 x 768 Pixel Hochformat / Querformat
			\end{itemize}
			\item \textbf{Maximale Bildschirmauflösung :}
			\begin{itemize}
				\item 4096 × 2160 Pixel Hochformat / Querformat
			\end{itemize}
				
		\end{itemize}
		
		
	
	\end{itemize}
	\subsection{Orgware}
	\begin{itemize}
		\item Der Client benötigt eine Internetverbindung.
		\item Um eine befriedigende Nutzererfahrung zu gewährleisten, werden folgende Bandbreiten-Untergrenzen definiert:
		\begin{itemize}
			\item \textbf{ Desktop PC :}
			\begin{itemize}
				\item DSL Verbindung mit min. 2 Mbit/s Download-Bandbreite
			\end{itemize}
		\end{itemize}
	\end{itemize}
	\subsection{Produkt-Schnittstellen}
	
	
	\section{\Large SPEZIELLE ANFORDERUNGEN AN DIE ENTWICKLUNGS-UMGEBUNG}
	Entwicklung- und Testumgebung des Frontends: Siehe 10 Technische Produktentwicklung 
	\subsection{Software}
	\subsection{Hardware}
	\subsection{Orgware}
	\subsection{Entwicklungsschnittstellen}
	
	
	\section{\Large GLIEDERUNG IN TEILPRODUKTE}
	
	
	\section{\Large ERGÄNZUNGEN}
	Ein erster Testbetrieb wird in einer virtuellen Umgebung stattfinden. Dort wird dann zunächst ausgiebig die Stabilität und Sicherheit des Systems getestet.
	
	
	\section{\Large GLOSSAR}
	
	In diesem Kapitel wird die spezifische Sprache des Auftraggebers wie \textbf{ Kürzel } und \textbf{ Fachbegriffe } beschrieben, z.B. :
	\begin{itemize}
		\item \textbf{ Benutzer }
		\begin{itemize}
			\item Eine eindeutig (über Email und Passwort) identifizierbare Einheit, die sich mit dem System verbindet und ein Profil erstellt und selbständig verwaltet.
		\end{itemize}
	\end{itemize}
	\begin{itemize}
		\item \textbf{ Administrator }
		\begin{itemize}
			\item Bearbeitet Meldungen, die von einem Profil erzeugt wurden und handelt entsprechend der Richtlinien. Der Administrator hat je nach Sachlage der Meldung die Möglichkeit, Den Benutzer zu sperren oder zu ahnden.
		\end{itemize}
	\end{itemize}	